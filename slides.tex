\documentclass[12pt,a4paper]{beamer}
\usepackage[utf8x]{inputenc}
\usepackage{ucs}
\usepackage[english]{babel}
%\usepackage[german]{babel}
\usepackage{amsmath}
\usepackage{amsfonts}
\usepackage{amssymb}
\author{Michael Haas, haas@cl.uni-heidelberg.de}
\title{Spatial Ontologies}
\subtitle{Seminar: Raum-, Zeit- und Ereignisrepräsentation für die semantische Verarbeitung (Dr. Michael Herweg)}
\date{1-07-2013}
\newcommand{\tuple}[1]{\ensuremath{\left \langle #1 \right \rangle }}
\newcommand{\setof}[1]{\ensuremath{\left \{ #1 \right \}}}

\begin{document}


\begin{frame}
\maketitle
\end{frame}

\begin{frame}{Übersicht}
\begin{itemize}
\item Grundlagen Ontologie
\item Spatial Ontologies
\item Frames of Reference
\item Verfügbare Ontologien
\begin{itemize}
    \item SUMO
    \item DOLCE
    \item ...
\end{itemize}
\item Zusammenfassung 
\item \textbf{Fragen? Zu schnell? Fragen!}
\end{itemize}
\end{frame}


\begin{frame}{Motivationsfragen Ontologie}
 - How do we share knowledge? etc TODO
\end{frame}





\begin{frame}{Definition Ontology}
\begin{quote}
An ontology is an explicit specification of a conceptualization.
\end{quote}
Gruber, 1993
\end{frame}

\begin{frame}{Definition Ontology}
Auf Deutsch: TODO
\end{frame}


Constructing a view of spatial semantics as an additional layer of ontology in this way brings several advantages crucial
for adequately capturing the relationship between language use and spatial interpretation. First, it supports the application
of the full range of methods developed within ontological engineering and applied ontology in order to organize the in-
formation necessary in ways that conform to a strict and formally specified modeling style [65,67,148]. Second, it provides
a suitable level of abstraction for dealing effectively with spatial language and for describing what linguistic expressions
themselves bring to the interpretation process—something that has not been found possible when focusing on linguistic ele-
ments as isolated terms. And third, it allows the relationship between linguistic expressions and spatial interpretation to be
recast as a particular case of ontological alignment, or mediation, whereby two or more distinct ontologies are brought into
a formal relationship [89,100,99,80,98]. Combining these considerations establishes a formally robust and well grounded
framework from which to consider the full flexibility required for dealing with the mapping between language and space.


\begin{frame}{Spatial Ontology}
\begin{itemize}
\item Ontologie für räumliche Konzepte und Begriffe
\item Unterstützt:
\begin{itemize}
    \item Pfadbeschreibung
    \item Szenenbeschreibung
    \item Navigation
\end{itemize}
\item Einsatz in NLP
\begin{itemize}
    \item Schnittstelle zu GIS
    \item Kontext-Basierte Dienste (??)
    \item Roboter
    \item Generell: Abbildung Raum-Sprache
\end{itemize}
\end{itemize}
\end{frame}


%We propose for this a linguistically-motivated ontology, or ‘linguistic ontology’ for short, that provides a specification of an
%encapsulated layer of ontological information motivated solely by the requirements of linguistically-expressed spatial mean-
%ings. This provides a new layer of organization for capturing the contribution of language to spatial interpretation that is
%free of non-linguistic, contextually-dependent additions. We thereby decompose and modularize the problem of interpreting
%and producing spatial language by ‘stratifying’ three ways: (i) lexicogrammatically, (ii) according to a shallow semantics,
%and (iii) by contextualized specifications—all three of which are formally distinct. The application of ontological engineering
%methods then offers considerable benefits for teasing apart the respective contributions made by these essentially distinct
%knowledge sources.



\begin{frame}{Linguistisch motivierte räumliche Ontologie}
\begin{itemize}
\item Verarbeitung räumlicher Ausdrücke benötigt formalisiertes Wissen über mögliche Ausdrücke
\item Ontologie muss sich aus sprachlicher Evidenz ergeben
\item Formalisierung benötigt systematische Betrachtung sprachlicher Ausdrücke
\item $\to$ Empirisch motiviert
\end{itemize}

\end{frame}





\begin{frame}{}
\begin{itemize}
\item
\end{itemize}

\end{frame}



%particularly, cognitive modelling, artificial intelligence, geographic information systems (GIS),
%and the like. The role of an ontology in this area is as it is in all domains and as was introduced
%in our baseline ontology deliverable D1: that is, (i) to set out a consistent and well-specified
%general modelling scheme which is free of contradiction and from which follows a set of generic
%properties that necessarily hold over the entities covered and, (ii) to support problem solving
%and inference within the domain of concern.



%\begin{frame}{CMSM for Sentiment Analysis: Eval Results}
%\begin{figure}
%\includegraphics[scale=0.45]{cmsm_for_se_table3.png}
%\caption{Yessenalina \&  Cardie (2011)}
%\end{figure}

%\end{frame}


\begin{frame}[allowframebreaks]{References}
\begin{thebibliography}{-}
% APA
\bibitem{bateman2010} Bateman, J. A., Hois, J., Ross, R., \& Tenbrink, T. (2010). A linguistic ontology of space for natural language processing. Artificial Intelligence, 174(14), 1027-1071.
\bibitem{bateman2004} Bateman, J., \& Farrar, S. (2004). Spatial ontology baseline. Collaborative Research Center for Spatial Cognition. I1-[OntoSpace] D2
\bibitem{gruber1993} Thomas R. Gruber. A Translation Approach to Portable Ontology Specifications. Knowledge Acquisition, 5(2):199-220, 1993.
\end{thebibliography}
\end{frame}
\end{document}

