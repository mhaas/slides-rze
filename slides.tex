\documentclass[12pt,a4paper]{beamer}
\usepackage[utf8x]{inputenc}
\usepackage{ucs}
\usepackage[english]{babel}
%\usepackage[german]{babel}
\usepackage{amsmath}
\usepackage{amsfonts}
\usepackage{amssymb}
\author{Michael Haas, haas@cl.uni-heidelberg.de}
\title{Spatial Ontologies}
\subtitle{Seminar: Raum-, Zeit- und Ereignisrepräsentation für die semantische Verarbeitung (Dr. Michael Herweg)}
\date{1-07-2013}
\newcommand{\tuple}[1]{\ensuremath{\left \langle #1 \right \rangle }}
\newcommand{\setof}[1]{\ensuremath{\left \{ #1 \right \}}}

\begin{document}


\begin{frame}
\maketitle
\end{frame}

\begin{frame}{Übersicht}
\begin{itemize}
\item Grundlagen Ontologie
\item Spatial Ontologies
\item Frames of Reference
\item Verfügbare Ontologien
\begin{itemize}
    \item SUMO
    \item DOLCE
    \item ...
\end{itemize}
\item Zusammenfassung 
\item \textbf{Fragen? Zu schnell? Fragen!}
\end{itemize}
\end{frame}


\begin{frame}{Motivationsfragen Ontologie}
 - How do we share knowledge? etc TODO
\end{frame}





\begin{frame}{Definition Ontology}
\begin{quote}
An ontology is an explicit specification of a conceptualization.
\end{quote}
Gruber, 1993
\end{frame}

\begin{frame}{Definition Ontology}
Auf Deutsch: TODO
\end{frame}


%Constructing a view of spatial semantics as an additional layer of ontology in this way brings several advantages crucial
%for adequately capturing the relationship between language use and spatial interpretation. First, it supports the application
%of the full range of methods developed within ontological engineering and applied ontology in order to organize the in-
%formation necessary in ways that conform to a strict and formally specified modeling style [65,67,148]. Second, it provides
%a suitable level of abstraction for dealing effectively with spatial language and for describing what linguistic expressions
%themselves bring to the interpretation process—something that has not been found possible when focusing on linguistic ele-
%ments as isolated terms. And third, it allows the relationship between linguistic expressions and spatial interpretation to be
%recast as a particular case of ontological alignment, or mediation, whereby two or more distinct ontologies are brought into
%a formal relationship [89,100,99,80,98]. Combining these considerations establishes a formally robust and well grounded
%framework from which to consider the full flexibility required for dealing with the mapping between language and space.


\begin{frame}{Spatial Ontology}
\begin{itemize}
\item Ontologie für räumliche Konzepte und Begriffe
\item Unterstützt:
\begin{itemize}
    \item Pfadbeschreibung
    \item Szenenbeschreibung
    \item Navigation
\end{itemize}
\item Einsatz in NLP
\begin{itemize}
    \item Schnittstelle zu GIS
    \item Kontext-Basierte Dienste (??)
    \item Roboter
    \item Generell: Abbildung Raum-Sprache
\end{itemize}
\end{itemize}
\end{frame}


%We propose for this a linguistically-motivated ontology, or ‘linguistic ontology’ for short, that provides a specification of an
%encapsulated layer of ontological information motivated solely by the requirements of linguistically-expressed spatial mean-
%ings. This provides a new layer of organization for capturing the contribution of language to spatial interpretation that is
%free of non-linguistic, contextually-dependent additions. We thereby decompose and modularize the problem of interpreting
%and producing spatial language by ‘stratifying’ three ways: (i) lexicogrammatically, (ii) according to a shallow semantics,
%and (iii) by contextualized specifications—all three of which are formally distinct. The application of ontological engineering
%methods then offers considerable benefits for teasing apart the respective contributions made by these essentially distinct
%knowledge sources.



\begin{frame}{Linguistisch motivierte räumliche Ontologie}
\begin{itemize}
\item Verarbeitung räumlicher Ausdrücke benötigt formalisiertes Wissen über mögliche Ausdrücke
\item Ontologie muss sich aus sprachlicher Evidenz ergeben
\item Formalisierung benötigt systematische Betrachtung sprachlicher Ausdrücke
\item $\to$ Empirisch motiviert
\end{itemize}
\end{frame}


% FOR: Bewegung
\begin{frame}{Frames of Reference}
\begin{itemize}
\item Räumliche Ontologie muss diverse sprachliche Phänomene abbilden
\item Frames of Reference: Welches Bezugssystem ist relevant?
\end{itemize}

Ein Boot ist in einem fließenden Fluss verankert. Bewegt sich das Boot?

\end{frame}

% FOR: intrinsische VS extrinsische orientierung
\begin{frame}{Frames of Reference}

\begin{figure}
\includegraphics[scale=0.45]{img/levinson_fig_2-1.png}
\caption{Wo ist der Ball? (Figure 2.1, Levinson 2003)}
\end{figure}
\end{frame}


\begin{frame}{Absolute VS relative Frame of Reference}
\begin{itemize}
\item Europäische Sprachen mit relativem Bezugssystem \\
\textit{Is the hot water in the right tap?}
\item Tzeltal mit absolutem Bezugssystem \\
\textit{Is the hot water in the uphill tap?} \\
$\to$ \textit{Is the hot water in the tap that would lie in the uphill southerly direction if I were at home?}
\end{itemize}
\end{frame}



\begin{frame}{Intrinsic Frame of Reference}
\begin{figure}
\includegraphics[scale=0.45]{img/levinson_FOR_intrinsic.png}
\caption{Intrinsic FoR (Levinson 2003)}
\end{figure}
\end{frame}

\begin{frame}{Intrinsic Frame of Reference}
\begin{itemize}
\item Im Englischen: Orientierung ergibt sich aus funktionalen Aspekten
\item In Tzeltal: basiert auf Form des Objektes
%\item Oft: Terminologie abgeleitet von Körperteilen\\
%\textit{Am Fuße des Berges}
\item Oft: nur ein Punkt erforderlich; "vorne" ergibt "hinten" \\
$\to$ Nicht in allen Sprachen derartige Opposition verfügbar
\item Suchdomäne unterschiedlich: wie groß ist "vorne"? \\
$\to$ \textit{In front of the church lie the mountains\ldots}
\end{itemize}
\end{frame}


\begin{frame}{Relative Frame of Reference}
\begin{figure}
\includegraphics[scale=0.45]{img/levinson_FOR_relative.png}
\caption{Relative FoR (Levinson 2003)}
\end{figure}
\end{frame}

\begin{frame}{Relative Frame of Reference}
\begin{itemize}
\item Koordinationsystem ausgehend von aktuellem Standpunkt V
\item Projiziert auf relatum G
\item Standpunkt muss nicht eigener Ort sein \\
$\to$ \textit{Bill kicked the ball to the left of the goal.}
\item Projektion kann Rotation oder Translation erfahren
% ?? \item Nicht alle Sprachen habe relative FoR ausserhalb intrinsischem Bezug!
\item Ambiguität zwischen intrinsischem und relativem FoR durch überlappende Begriffe \\
$\to$ \textit{to the left of the chair} VS \textit{at the chair's left}
\item Manche Sprachen benutzen "links" etc nur intrinsisch
\item Relative FoR: verzichtbar
\item Kinder lernen projektives "links" erst spät
\item Einige Sprachen benutzen relativen FoR gar nicht
\end{itemize}
\end{frame}


\begin{frame}{Absolute Frame of Reference}
\begin{figure}
\includegraphics[scale=0.45]{img/levinson_FOR_absolute.png}
\caption{Absolute FoR (Levinson 2003)}
\end{figure}
\end{frame}

\begin{frame}{Absolute Frame of Reference}
\begin{itemize}
\item Kein direkter Bezug zu Orientierung von Betrachter oder Objekt
\item Vertikal: an Schwerkraft orientiert
\item Horizontal: Himmelsrichtungen, landschaftliche Referenzpunkte
\item Sprecher müssen Orientierung behalten
\item Geschätzt ein Drittel aller Sprachen
% Später? \item Unterstützen transitive Inferenz
\end{itemize}
\end{frame}


\begin{frame}{Absolute Frame of Reference: Achsen}
\begin{itemize}
\item Europa: Norden, Süden, Westen, Osten
\item Tenejapa: Norden, Süden, Quer
\item Bali: Achsen gegeben von Monsun und Berg \\
$\to$ Orthogonalität?
\end{itemize}
\end{frame}


\begin{frame}{Drei Frames of Reference}
\begin{itemize}
\item Intrinsic
\item Relative
\item Absolute
\end{itemize}
\end{frame}


\begin{frame}{Frames of Reference: Inferenz und Eigenschaften}
\begin{itemize}
\item Weinheim liegt nördlich von Heidelberg. Darmstadt liegt nördlich von Weinheim. \\
$\to$ Darmstadt liegt nördlich von Heidelberg.
\item Weinheim liegt nördlich von Heidelberg. \\
$\to$ Heidelberg liegt südlich von Weinheim.
\item Jill ist links von Jack. Bill ist links von Jill. \\
$\to$ ?? Bill ist links von Jack ??
\end{itemize}
\end{frame}


\begin{frame}{Transitivity und Converseness}
\begin{figure}
\includegraphics[scale=0.45]{img/levinson_FOR_transitivity.png}
\caption{Transitivity and Converseness (Levinson 2003)}
\end{figure}
Transitivity: Jill ist links von Jack. Bill ist links von Jill. \\
\textbf{!} Bill ist links von jack.\\
Converseness: Jill ist links von Jack. \\
\textbf{!} Jack ist rechts von Jill.
\end{frame}



\begin{frame}{Transitivity und Converseness}
\begin{itemize}
\item Funktioniert für absolute Frame of Reference
\item Funktioniert für relative Frame of Reference bei gleichem Standpunkt
\item Funktioniert \textbf{nicht} für intrinsic Frame of Reference
\end{itemize}
\end{frame}


\begin{frame}{Ontologien}
\begin{itemize}
\item Welche (räumlichen) Ontologien gibt es?
\item Wie berücksichtigen Ontologien die Frames of Reference?
% Müssen Ontologien alle FoR berücksichtigen, oder nur für NLP?
% TODO: Konvertierbarkeit FoR?
\end{itemize}
\end{frame}


\begin{frame}{SUMO}
\begin{itemize}
\item SUMO: Suggested Upper Merged Ontology
\item Kategorien für physische Objekte
\item Mereotopologie
\item Räumliche Relationen
\end{itemize}
\end{frame}

\begin{frame}{SUMO: Object}
(<=> (instance ?PHYS Physical)
(exists (?LOC ?TIME)
(and
(located ?PHYS ?LOC)
(time ?PHYS ?TIME))))
%\includegraphics[scale=0.45]{img/d2_object_axiom.png}
%\caption{SUMO Object axiom (Bateman et al, 2004)}
%\end{figure}
%\end{itemize}
\end{frame}




\begin{frame}{SUMO: Region}
%\begin{itemize}
(=>
(instance ?REGION Region)
(exists (?PHYS)
(located ?PHYS ?REGION)))
%\end{itemize}
\end{frame}



\begin{frame}{SUMO: Region Taxonomy}
%\begin{itemize}
\begin{figure}
\includegraphics[scale=0.45]{img/d2_region_taxonomy.png}
\caption{SUMO Object axiom (Bateman et al, 2004)}
\end{figure}
%\end{itemize}
\end{frame}

\begin{frame}{SUMO: Position}
PositionalAttribute
Vertical(i)
Horizontal(i)
Above(i)
Below(i)
Adjacent(i)
Left(i)
Right(i)
Near(i)
On(i)

Subclass: DirectionalAttribute mit Instanzen North, South, East, West
\end{frame}

\begin{frame}{SUMO: Position axioms}
(<=> (orientation ?OBJ1 ?OBJ2 Vertical)
(orientation ?OBJ2 ?OBJ1 Vertical))


(<=>
 (orientation ?OBJ1 ?OBJ2 Right)
(orientation ?OBJ2 ?OBJ1 Left))
% TODO: google "SUMO frame of reference"
\end{frame}



\begin{frame}{SUMO: Frames of Reference}
\begin{itemize}
\item Positions-Axiome aus wissenschaftlicher Sicht sinnvoll
\item Links immer Gegensatz von rechts?
\item Keine Berücksichtigung unterschiedlicher Frames of Reference
\end{itemize}
% TODO: google "SUMO frame of reference"
\end{frame}


\begin{frame}{OpenCyc}
\begin{quote}
The OpenCyc Platform is your gateway to the full power of Cyc, the world's largest and most complete general knowledge base and commonsense reasoning engine. \footnote{http://www.cyc.com/platform/opencyc Accessed 2013-06-30}
\end{quote}
\begin{figure}
\includegraphics[scale=0.45]{img/d2_opencyc_SpatialThing_taxonomy.png}
\caption{OpenCyc SpatialThing taxonomy (Bateman et al, 2004)}
\end{figure}
\end{frame}

\begin{frame}{OpenCyc: Sides of objects }
\begin{figure}
\includegraphics[scale=0.45]{img/d2_opencyc_sides_of_objects.png}
\caption{OpenCyc Sides of objects (Bateman et al, 2004)}
\end{figure}
\end{frame}

\begin{frame}{OpenCyc: Spatial Relations}
near[NaivePhysicsVocabularyMt]:
inFrontOf-Generally
behind-Generally
alignedAlong
hasPortalToRegion
movesWith
stuckTo
spatiallyIntersects
touches

\end{frame}

Does inFrontOf-Generally presuppose Side surface?


\begin{frame}{OpenCyc: Orientation \& Direction Attributes}
\begin{itemize}
\item Orientierung
HorizontalOrientation
VerticalOrientation
UpsideDown
RightSideUp

\item Richtung
Up-Generally
Up-Directly
Down-Generally
Down-Directly
VerticalDirection
HorizontalDirection


\end{itemize}
\end{frame}



\begin{frame}{OpenCyc: Geographische Richtung}
North-Generally
North-Directly
South-Directly
...
East-Directly
%\begin{itemize}
%\end{itemize}
\end{frame}


\begin{frame}{OpenCyc: Issues}
\begin{itemize}
\item Keine Beachtung von Frames of Reference
\item Reflektiert teils Besonderheiten des Englischen
$\to$ \textit{in-Floating} - Bewegung oder Attribut?
\end{itemize}
\end{frame}



\begin{frame}{GUM: Generalized Upper Model}
\begin{itemize}
\item Räumliche Erweiterung: basiert rein auf linguistischer Evidenz
\end{itemize}
\end{frame}


\begin{frame}{GUM: Beispiel}
To the left of the computer is a USB drive.

\begin{figure}
\includegraphics[scale=0.45]{img/2010_fig3.png}
\caption{GUM Representation(Bateman et al, 2010)}
\end{figure}
\end{frame}


\begin{frame}{GUM: Beispiel}
To the left of the computer is a USB drive.

\begin{figure}
\includegraphics[scale=0.45]{img/2010_gl_of_usb.png}
\caption{GUM Representation(Bateman et al, 2010)}
\end{figure}
\end{frame}


\begin{frame}{GUM: Bewegung, Orientierung ohne Zustandsänderung}
$$ NonAffectingOrienting  $$
$$ NonAffectingMotion \equiv NonAffectingDirectedMotion \sqcup NonAffectingOrientationChange \sqcup NonAffectingSimpleMotion$$
\begin{itemize}
\item $NonAffectingOrienting$: "We are facing the table"
\item $NonAffectingSimpleMotion$:  "I ran", "The ball rolls"
\item $NonAffectingDirectedMotion$: "He walks forward"
\item $NonAffectingOrientationChange$: "He turned around"
\end{itemize}
\end{frame}


\begin{frame}{GUM: Bewegung mit Zustandsänderung}
Dynamische Prozesse: "He puts the USB drive on the table"
$$AffectingAction \equiv DoingAndHappening \sqcap \exists actee.SimpleThing$$
$$AffectingSpatialAction$$
\begin{figure}
\includegraphics[scale=0.45]{img/2010_puts_usb_on_table.png}
\caption{GUM Representation(Bateman et al, 2010)}
\end{figure}
\end{frame}

\begin{frame}{GUM: Spatial Modalities}
$$  SpatialModality \equiv SpatialDistanceModality \sqcup FunctionalSpatialModality \sqcup RelativeSpatialModality $$
\begin{itemize}
\item \textbf{SpatialModality} füllt \textbf{hasSpatialModality} Relation in \textbf{GeneralizedLocations}
\begin{itemize}
    \item Statisch: zwischen \textbf{locatum} und \textbf{relatum}
    \item Dynamisch: zwischen \textbf{actor/actee} und \textbf{relatum}
\end{itemize}
\item Beispiele: Support, Proximal, Left-Of
\item Extrema: \textbf{extremePositionOnAxis}, \textbf{extremeDistancePosition} \\
$\to$ "the leftmost", "the closest"
\item Winkel und Distanzen: \textbf{quantitativeAngleExtent}, \textbf{quantitativeDistanceExtent} \\
$\to$ "They turn around 180 degrees"
\item Auch qualitativ: \textbf{qualitativeAngleExtent}, \textbf{qualitativeDistanceExtent} \\
$\to$ "Turn a bit further to your right"
\end{itemize}
\end{frame}


\begin{frame}{GUM: Relative Aspekte \& Projektionen}
\begin{itemize}
\item Subklasse \textbf{RelativeSpatialModality}
\end{itemize}
\end{frame}



\begin{frame}{GUM: Frames of Reference}
\begin{itemize}
\item Frames of Reference realisiert über \textbf{SpatialModality}
\item Konkreter Frame of Reference wird über Kontextualisierung eingefügt \\
$\to$ Ontologische Repräsentation hat noch keinen konkreten FoR
\end{itemize}
\end{frame}





%particularly, cognitive modelling, artificial intelligence, geographic information systems (GIS),
%and the like. The role of an ontology in this area is as it is in all domains and as was introduced
%in our baseline ontology deliverable D1: that is, (i) to set out a consistent and well-specified
%general modelling scheme which is free of contradiction and from which follows a set of generic
%properties that necessarily hold over the entities covered and, (ii) to support problem solving
%and inference within the domain of concern.



%\begin{frame}{CMSM for Sentiment Analysis: Eval Results}
%\begin{figure}
%\includegraphics[scale=0.45]{cmsm_for_se_table3.png}
%\caption{Yessenalina \&  Cardie (2011)}
%\end{figure}

%\end{frame}


\begin{frame}[allowframebreaks]{References}
\begin{thebibliography}{-}
% APA
\bibitem{bateman2010} Bateman, J. A., Hois, J., Ross, R., \& Tenbrink, T. (2010). A linguistic ontology of space for natural language processing. Artificial Intelligence, 174(14), 1027-1071.
\bibitem{bateman2004} Bateman, J., \& Farrar, S. (2004). Spatial ontology baseline. Collaborative Research Center for Spatial Cognition. I1-[OntoSpace] D2
\bibitem{gruber1993} Thomas R. Gruber. A Translation Approach to Portable Ontology Specifications. Knowledge Acquisition, 5(2):199-220, 1993.
\bibitem{levinson2003} Levinson, S. C. (2003). Space in language and cognition: Explorations in cognitive diversity (Vol. 5). Cambridge University Press.
\end{thebibliography}
\end{frame}
\end{document}

